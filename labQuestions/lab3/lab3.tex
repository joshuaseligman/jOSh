%%%%%%%%%%%%%%%%%%%%%%%%%%%%%%%%%%%%%%%%%
%
% CMPT 424
% Fall 2022
% Lab 3
%
%%%%%%%%%%%%%%%%%%%%%%%%%%%%%%%%%%%%%%%%%

%%%%%%%%%%%%%%%%%%%%%%%%%%%%%%%%%%%%%%%%%
% Short Sectioned Assignment
% LaTeX Template
% Version 1.0 (5/5/12)
%
% This template has been downloaded from: http://www.LaTeXTemplates.com
% Original author: % Frits Wenneker (http://www.howtotex.com)
% License: CC BY-NC-SA 3.0 (http://creativecommons.org/licenses/by-nc-sa/3.0/)
% Modified by Alan G. Labouseur  - alan@labouseur.com
%
%%%%%%%%%%%%%%%%%%%%%%%%%%%%%%%%%%%%%%%%%

%----------------------------------------------------------------------------------------
%	PACKAGES AND OTHER DOCUMENT CONFIGURATIONS
%----------------------------------------------------------------------------------------

\documentclass[letterpaper, 10pt,DIV=13]{scrartcl} 

\usepackage[T1]{fontenc} % Use 8-bit encoding that has 256 glyphs
\usepackage[english]{babel} % English language/hyphenation
\usepackage{amsmath,amsfonts,amsthm,xfrac} % Math packages
\usepackage{sectsty} % Allows customizing section commands
\usepackage{graphicx}
\usepackage[lined,linesnumbered,commentsnumbered]{algorithm2e}
\usepackage{listings}
\usepackage{parskip}
\usepackage{lastpage}

\allsectionsfont{\normalfont\scshape} % Make all section titles in default font and small caps.

\usepackage{fancyhdr} % Custom headers and footers
\pagestyle{fancyplain} % Makes all pages in the document conform to the custom headers and footers

\fancyhead{} % No page header - if you want one, create it in the same way as the footers below
\fancyfoot[L]{} % Empty left footer
\fancyfoot[C]{} % Empty center footer
\fancyfoot[R]{page \thepage\ of \pageref{LastPage}} % Page numbering for right footer

\renewcommand{\headrulewidth}{0pt} % Remove header underlines
\renewcommand{\footrulewidth}{0pt} % Remove footer underlines
\setlength{\headheight}{13.6pt} % Customize the height of the header

\numberwithin{equation}{section} % Number equations within sections (i.e. 1.1, 1.2, 2.1, 2.2 instead of 1, 2, 3, 4)
\numberwithin{figure}{section} % Number figures within sections (i.e. 1.1, 1.2, 2.1, 2.2 instead of 1, 2, 3, 4)
\numberwithin{table}{section} % Number tables within sections (i.e. 1.1, 1.2, 2.1, 2.2 instead of 1, 2, 3, 4)

\setlength\parindent{0pt} % Removes all indentation from paragraphs.

\binoppenalty=3000
\relpenalty=3000

%----------------------------------------------------------------------------------------
%	TITLE SECTION
%----------------------------------------------------------------------------------------

\newcommand{\horrule}[1]{\rule{\linewidth}{#1}} % Create horizontal rule command with 1 argument of height

\title{	
   \normalfont \normalsize 
   \textsc{CMPT 424 - Fall 2022 - Dr. Labouseur} \\[10pt] % Header stuff.
   \horrule{0.5pt} \\[0.25cm] 	% Top horizontal rule
   \huge Lab 3  \\     	    % Assignment title
   \horrule{0.5pt} \\[0.25cm] 	% Bottom horizontal rule
}

\author{Josh Seligman \\ \normalsize joshua.seligman1@marist.edu}

\date{\normalsize\today} 	% Today's date.

\begin{document}
\maketitle % Print the title

% Question 1
\section{Question 1}
\textit{Explain the difference between internal and external fragmentation.} \\ \\

% Question 2
\section{Question 2}
\textit{Given Five (5) memory partitions of 100KB, 500KB, 200KB, 300KB, and 600KB (in that order), how would optimal, first-fit, best-fit, and worst-fit algorithms place processes of 212KB, 417KB, 112KB, and 426KB (in that order)?} \\ \\
The first-fit algorithm places each process in the first available memory partition that can hold the process. Thus, the process of size 212KB will be placed in the partition of 500KB, the process of size 417KB will be placed in the partition of 600KB, and the process of size 112KB will be placed in the partition of 200KB. For the process of size 426KB, the partitions that can hold it (500KB and 600KB) are both in use, so the process will either have to wait for one of these partitions to open up or it will be rejected, depending on the OS.

Thie best-fit algorithm will place each process in the smallest available partition that can support the size of the process. Therefore, the process of size 212KB will be placed in the partition of size 300KB, the process of size 417KB will be placed in the partition of size 500KB, the process of size 112KB will be placed in the partition of size 200KB, and the process of size 426KB will be placed in the partition of size 600KB. In this scenario, no processes are rejected as they are all allocated a partition in memory without conflict.

The worst-fit algorithm places each process in the lagest available partition. Thus, the process of size 212KB will be placed in the partition of size 600KB, the partition of size 417KB will be placed in the partition of size 500KB, and the process of size 112KB will be placed in the partition of size 300KB. Similar to the first-fit algorithm, the process of size 426KB does not have an available partition to be stored in, so it will be rejected.

\end{document}
